\subsection{Programos veikimo pavyzdžiai}

\lstset{language=Java,keywordstyle={\bfseries \color{blue} \textbf}}

Atidarius programą matomas trumpas jos aprašymas. Programą galima lokalizuoti įvedus komandą \inCode{editorModel.locale = "lt_LT"} (priedas \ref{appendix:run_examples_welcome}). Komandų sąrašas (į „window“ objektą iškelti programos objektai) pateiktas \ref{tab:program_commands} lentelėje. Paspaudus mygtuką \uiWord{Naujas} matomas diagramos kūrimo langas (priedas \ref{appendix:run_examples_new}). Pasirinkus \uiWord{\DVCM{}} rodomas \DVCM{} kūrimo sąsaja (priedas \ref{appendix:run_examples_new_dvcm}). Joje galimos \DVCM{} redagavimo komandos.

\stepcounter{counter:table:reset}

\begin{center}
    \begin{longtable}{ | l | p{14cm} |}
    \caption{Demonstracinės programos konsolės komandų sąrašas.}
	  \label{tab:program_commands}

    \\ \hline
      \textbf{Nr.}
      &
      \rownumber
    \\ \hline
      \textbf{Komanda}
      &
      editorModel.locale = [lokalės\_kodas]
    \\ \hline
      \textbf{Paaiškinimas}
      &
      Lokalė pakeičiama į lokalės kodu nurodytą lokalę. Lokalė yra simbolių eilutės tipo ISO/IEC 15897 standarto. Nustatyta lokalė yra anglų, ją dar galima pakeisti į lietuvių.
    \\ \hline
      \textbf{Pavyzdžiai}
      &
      \inCode{editorModel.locale = "lt_LT"}
    \\ \hline

    \multicolumn{2}{c}{}

    \\ \hline
      \textbf{Nr.}
      &
      \rownumber
    \\ \hline
      \textbf{Komanda}
      &
      editorModel.diagram.selected.id
    \\ \hline
      \textbf{Paaiškinimas}
      &
      Rezultatas yra pažymėto komponento id.
    \\ \hline

    \multicolumn{2}{c}{}

    \\ \hline
      \textbf{Nr.}
      &
      \rownumber
    \\ \hline
      \textbf{Komanda}
      &
      editorModel.diagram.getComponent([id])
    \\ \hline
      \textbf{Paaiškinimas}
      &
      Rezultatas yra komponento objektas.
    \\ \hline
      \textbf{Pavyzdžiai}
      &
    \inCode{editorModel.diagram.getComponent(1)}
    \\ \hline

    \multicolumn{2}{c}{}



    \end{longtable}
\end{center}
