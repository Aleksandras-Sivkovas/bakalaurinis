\subsection{Programos veikimo pavyzdžiai}

\lstset{language=Java,keywordstyle={\bfseries \color{blue} \textbf}}

Atidarius programą matomas trumpas jos aprašymas. Programą galima lokalizuoti į konsolę įvedus komandą \inCode{editorModel.locale = "lt_LT"} (priedas \ref{appendix:run_examples_welcome}). Programa pateikia grafinę diagramų atvaizdavimo sąsają, bet jų kūrimui pateikiamas komandinis interfeisas. Komandomis vadinami „window“ objektui priskirti programos objektai (naudojimo pavyzdys: kodas \ref{lst:commands_to_create_dvcm}). Paspaudus mygtuką \uiWord{Naujas} matomas diagramos kūrimo langas (priedas \ref{appendix:run_examples_new}). Pasirinkus \uiWord{\DVCM{}} rodomas \DVCM{} kūrimo sąsaja (priedas \ref{appendix:run_examples_new_dvcm}). Joje galimos \DVCM{} redagavimo komandos. Suvedus komandas iš \ref{lst:commands_to_create_dvcm} kodo gaunamas \DVCM{} modelis. Šiek tiek pakoreguotas grafine sąsaja jis pavaizduotas priede \ref{appendix:run_examples_created_dvcm}. Sudėtingesnio \DVCM{} (priedas \ref{appendix:dvcm_window}) \JSON{} formato duomenų failas yra pateiktas \href{https://aleksandras-sivkovas.github.io/diagrams-editor-app/dvcm.json}{https://aleksandras-sivkovas.github.io/diagrams-editor-app/dvcm.json}.

\renewcommand{\lstlistingname}{Kodas}
\renewcommand{\lstlistlistingname}{Kodo fragmentai}
\setcounter{lstlisting}{0}
\lstinputlisting[style=javascript, caption={Komandos gauti \ref{appendix:run_examples_created_dvcm} priede pavaizduotą \DVCM}, label={lst:commands_to_create_dvcm}]{sections/prototype_app/code/commands_for_dvcm.js}
