\section{Programa BPMN transformacijai į užduočių diagramą}


Vienas iš šio darbo tikslų yra programa, demonstruojanti algoritmo \BPMN{} transformacijai į užduočių diagramą, veikimą. Programos įeigai pasirinktas \BPMN{} išplėtimas \DVCM{} modelis, išeiga – užduočių diagrama. Algoritmas aprašytas \ref{section:main_use_cases_from_bpmn} skyriuje. Programos realizacija yra „github“ sistemoje. Ten galima rasti:
\begin{enumerate}
  \item Programos kodą: \href{https://github.com/Aleksandras-Sivkovas/diagrams-editor-app}{https://github.com/Aleksandras-Sivkovas/diagrams-editor-app}.
  \item Sutransliuotą demo versiją: \href{https://aleksandras-sivkovas.github.io/diagrams-editor-app/}{https://aleksandras-sivkovas.github.io/diagrams-editor-app/}.
  \item Pavyzdinį \JSON{} formato duomenų failą, kurį galima importuoti programoje arba generuoti iš jo užduočių diagramas: \href{https://aleksandras-sivkovas.github.io/diagrams-editor-app/dvcm.json}{https://aleksandras-sivkovas.github.io/diagrams-editor-app/dvcm.json}.
\end{enumerate}
Algoritmo aprašyto \ref{section:main_use_cases_from_bpmn} skyriuje realizaciją „Javascript“ kalba pateikiama \ref{appendix:pseudocode_implementation} priede.

\subsection{Funkciniai reikalavimai programai}

Čia aprašomi demonstracinės programos funkciniai reikalavimai. Jie pateikiami užduočių diagramos pavidalu (\ref{img:program_functional_requirements} pav.) bei aprašomi \ref{tab:program_re_dvcm_creation}, \ref{tab:program_re_dvcm_save} ir \ref{tab:program_re_uc_generation} lentelėse. Šie reikalavimai parodys kokiomis funkcijos galės pasinaudoti naudotojas norėdamas pasižiūrėti algoritmo veikimą.

\begin{figure}[H]
	\centering
	\includegraphics[width=\textwidth]{sections/prototype_app/img/program_functional_requirements}
	\caption{Demonstracinės programos funkciniai reikalavimai.}
	\label{img:program_functional_requirements}
\end{figure}

\begin{center}
    \begin{longtable}{|p{\textwidth}|}
    \caption{\DVCM{} kūrimo naudojimo atvejis}
	\label{tab:program_re_dvcm_creation}
	\\ \hline
    \begin{tabular}{@{}p{3.5cm}p{12cm}}
    	\\
    	\textbf{ID} & NA1
    	\\
    	\textbf{Pavadinimas} & \DVCM{} sukūrimas
    	\\
    \end{tabular}
    \\
    \textbf{Aktoriai}
    \begin{enumerate}
    	\item Naudotojas.
	\end{enumerate}
    \\
    \textbf{Aprašas}

      Naudotojas pasinaudojęs programa sukuria \DVCM{}.

    \\
    \textbf{Prieš sąlygos}
    \begin{enumerate}
    	\item Naudotojas yra atidaręs programos langą.
	\end{enumerate}
    \\
    \textbf{Priežastys}
    \begin{enumerate}
    	\item Naudotojas pareikalavo sukurti \DVCM{}.
	\end{enumerate}
    \\
    \textbf{Po sąlygos}
    \begin{enumerate}
    	\item Naudotojas yra sukūręs \DVCM{}.
      \item Naudotojas mato savo sukurtą \DVCM{} programos lange.
	\end{enumerate}
    \\
    \textbf{Pagrindinė užduočių seka}
    \begin{enumerate}
    	\item Sistema pateikia interfeisą \DVCM{} kūrimui.
    	\item Naudotojas kuria \DVCM{}.
	\end{enumerate}
    \\
    \textbf{Alternatyvios užduočių sekos}

    \\
    \textbf{Išimtinės užduočių sekos}

    \\
    \\ \hline
    \end{longtable}
\end{center}

\begin{center}
    \begin{longtable}{|p{\textwidth}|}
    \caption{\DVCM{} saugojimo naudojimo atvejis}
	\label{tab:program_re_dvcm_save}
	\\ \hline
    \begin{tabular}{@{}p{3.5cm}p{12cm}}
    	\\
    	\textbf{ID} & NA2
    	\\
    	\textbf{Pavadinimas} & \DVCM{} išsaugojimas
    	\\
    \end{tabular}
    \\
    \textbf{Aktoriai}
    \begin{enumerate}
    	\item Naudotojas.
	\end{enumerate}
    \\
    \textbf{Aprašas}

      Naudotojas išsaugo programna sukurtą \DVCM{}.

    \\
    \textbf{Prieš sąlygos}
    \begin{enumerate}
    	\item Naudotojas yra atidaręs \DVCM{} kūrimo interfeisą.
	\end{enumerate}
    \\
    \textbf{Priežastys}
    \begin{enumerate}
    	\item Naudotojas pareikalavo išsaugoti \DVCM{}.
	\end{enumerate}
    \\
    \textbf{Po sąlygos}
    \begin{enumerate}
    	\item Naudotojas yra išsaugojęs \DVCM{}.
      \item Naudotojas žino kad \DVCM{} išsaugotas.
	\end{enumerate}
    \\
    \textbf{Pagrindinė užduočių seka}
    \begin{enumerate}
    	\item Demonstracinė programa pateikia išsaugojimo interfeisą.
    	\item Naudotojas pasirenka kaip saugoti \DVCM{}.
      \item Demonstracinė programa išsaugo \DVCM{}.
      \item Demonstracinė programa patvirtina, kad \DVCM{} išsaugotas.
	\end{enumerate}
    \\
    \textbf{Alternatyvios užduočių sekos}

    \\
    \textbf{Išimtinės užduočių sekos}
      TODO: cancel
    \\
    \\ \hline
    \end{longtable}
\end{center}

\begin{center}
    \begin{longtable}{|p{\textwidth}|}
    \caption{Užduočių diagramos generavimo pagal pateiktą \DVCM{} modelį naudojimo atvejis}
	\label{tab:program_re_uc_generation}
	\\ \hline
    \begin{tabular}{@{}p{3.5cm}p{12cm}}
    	\\
    	\textbf{ID} & NA3
    	\\
    	\textbf{Pavadinimas} & Užduočių diagramos generavimas pagal pateiktą \DVCM{}
    	\\
    \end{tabular}
    \\
    \textbf{Aktoriai}
    \begin{enumerate}
    	\item Naudotojas.
	\end{enumerate}
    \\
    \textbf{Aprašas}

      Naudotojas pasinaudojęs programa pateikia \DVCM{} ir programa jam sugeneruoja užduočių diagramą.

    \\
    \textbf{Prieš sąlygos}
    \begin{enumerate}
    	\item Naudotojas yra atidaręs programos langą.
	\end{enumerate}
    \\
    \textbf{Priežastys}
    \begin{enumerate}
    	\item Naudotojas pareikalavo sugeneruoti užduočių diagramą iš \DVCM{}.
	\end{enumerate}
    \\
    \textbf{Po sąlygos}
    \begin{enumerate}
    	\item Demonstracinėje programoje yra užduočių diagramos duomenys.
      \item Naudotojas mato sugeneruotą užduočių diagramą Demonstracinės programos lange.
	\end{enumerate}
    \\
    \textbf{Pagrindinė užduočių seka}
    \begin{enumerate}
      \item Demonstracinė programa pateikia užduočių diagramos generavimo iš \DVCM{} variantus.
      \item Naudotojas išsirenka generuoti bendrą diagramą iš \DVCM{}.
    	\item Demonstracinė programa pateikia interfeisą \DVCM{} įvedimui.
    	\item Naudotojas įveda \DVCM{}.
      \item Demonstracinė programa sugeneruoja užduočių diagramos modelį.
      \item Demonstracinė programa pavaizduoja užduočių diagramą savo lange.
	\end{enumerate}
    \\
    \textbf{Alternatyvios užduočių sekos}
      TODO: display by transaction
    \\
    \textbf{Išimtinės užduočių sekos}
      TODO: cancel
    \\
    \\ \hline
    \end{longtable}
\end{center}

\input{./sections/prototype_app/sections/technologies_used.tex}
\subsection{Programos veikimo pavyzdžiai}

\lstset{language=Java,keywordstyle={\bfseries \color{blue} \textbf}}

Atidarius programą matomas trumpas jos aprašymas. Programą galima lokalizuoti į konsolę įvedus komandą \inCode{editorModel.locale = "lt_LT"} (priedas \ref{appendix:run_examples_welcome}). Programa pateikia grafinę diagramų atvaizdavimo sąsają, bet jų kūrimui pateikiamas komandinis interfeisas. Komandomis vadinami „window“ objektui priskirti programos objektai (naudojimo pavyzdys: kodas \ref{lst:commands_to_create_dvcm}). Paspaudus mygtuką \uiWord{Naujas} matomas diagramos kūrimo langas (priedas \ref{appendix:run_examples_new}). Pasirinkus \uiWord{\DVCM{}} rodomas \DVCM{} kūrimo sąsaja (priedas \ref{appendix:run_examples_new_dvcm}). Joje galimos \DVCM{} redagavimo komandos. Suvedus komandas iš \ref{lst:commands_to_create_dvcm} kodo gaunamas \DVCM{} modelis. Šiek tiek pakoreguotas grafine sąsaja jis pavaizduotas priede \ref{appendix:run_examples_created_dvcm}. Sudėtingesnio \DVCM{} (priedas \ref{appendix:dvcm_window}) \JSON{} formato duomenų failas yra pateiktas \href{https://aleksandras-sivkovas.github.io/diagrams-editor-app/dvcm.json}{https://aleksandras-sivkovas.github.io/diagrams-editor-app/dvcm.json}.

\renewcommand{\lstlistingname}{Kodas}
\renewcommand{\lstlistlistingname}{Kodo fragmentai}
\setcounter{lstlisting}{0}
\lstinputlisting[style=javascript, caption={Komandos gauti \ref{appendix:run_examples_created_dvcm} priede pavaizduotą \DVCM}, label={lst:commands_to_create_dvcm}]{sections/prototype_app/code/commands_for_dvcm.js}

